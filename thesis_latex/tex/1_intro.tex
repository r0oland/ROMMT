%The Introduction section clarifies the motivation for the work presented and 
%prepares readers for the structure of the paper.
% context:  orient those readers who are less familiar with your topic and to 
%establish the importance of your work
% need: state the need for your work, as an opposition between what the 
%scientific community currently has and what it wants.
% task: indicate what you have done in an effort to address the need (this is 
%the task)
% object: preview the remainder of the paper to mentally prepare readers for 
%its structure, in the object of the document
%%%%%%%%%%%%%%%%%%%%%%%%%%%%%%%%%%%%%%%%%%%%%%%%%%%%%%%%%%%%%%%%%%%%%%%%%%%%%%%
\section{Introduction}
\label{sec:Introduction}
%%%%%%%%%%%%%%%%%%%%%%%%%%%%%%%%%%%%%%%%%%%%%%%%%%%%%%%%%%%%%%%%%%%%%%%%%%%%%%%

% General about standard microscopy
By magnifying minuscule cellular and subcellular features, optical microscopes provide a powerful tool for studying tissue components and their dynamic interactions. Its excellent imaging contrast in soft tissue has made optical microscopy the most widely used imaging modality in the biomedical community.\cite{2000_Amos_Lessonsfromhistory}  

The visual power of optical microscopy relies on sharp optical focusing. Such power is rapidly reduced as photons travel deeper into biological tissue, a highly scattering medium for electromagnetic waves in the optical spec- tral range. When photons reach the optical diffusion limit ($\approx \unit[1]{mm}$ in tissue), they have typically undergone tens of scattering events, which randomize the photon paths and thus prevent tight focusing~\cite{2000_Fujimoto_Opticalcoherencetomography:}

Although modern optical microscopic techniques have released biologists from the con?nes of ten-micrometer-thick ex vivo tissue slices to a world of volumetric in vivo tissue, optical microscopy is still challenged to image at depths beyond the optical diffusion limit while maintaining high resolution. For decades, engineers have made scant progress by using pure optical approaches to light scattering. 

When a short laser pulse, typically in the nanosecond range, is spatially broadened and then used to irradiate biological tissue, it produces a temperature rise on the order of milli-Kelvin in a short time frame. Consequently, thermoelastic expansion causes emission of acoustic waves, referred to as photoacoustic waves, that can be measured by wideband ultrasonic transducers around the sample. This phenomenon, discovered by Alexander Graham Bell, has been recently exploited for small-animal imaging, because the acquired photoacoustic waves can be combined mathematically to reconstruct the distribution of optical energy absorption. \cite{2005_Ntziachristos_Lookingandlistening}
 
Mesoscopy instead aims to a balance between penetration and resolution with poten- tial applications ranging from imaging structures of a few millimeter dimensions, such as microvasculature, or biological organisms, such as embryos, zebrafish, and drosophila.\cite{2013_Omar_Rasterscanoptoacoustic}  

\textbf{advantages}:
\begin{itemize}
	\item combining ultrasonic-scale spatial resolution with high sensitivity to tissue light absorption
\end{itemize}

\textbf{disadvantages/problems:}
\begin{itemize}
	\item characterized by long acquisition times and is generally not suitable for real time imaging of dynamic processes
\end{itemize}

\textbf{applications (so far for PA tomography}
\begin{itemize}
	\item visualization of the brain structure and lesions, of cerebral hemodynamic responses to hyperoxia and hypoxia and of cerebral cortical responses to neuroactivities induced by whisker stimulations in rats \cite{2003_Wang_Noninvasivelaserinduced}
	\item noninvasive in vivo imaging of exogenous contrast agents in the rat brain using indocyanine green stabilized with polyethylene glycol \cite{2004_Wang_Noninvasivephotoacousticangiography} 
	\item can be applied to different biomedical research areas, including cancer, cardiovascular, immunologic/inflammatory and neurodegenerative diseases
	\item detection of abnormal capillary shapes and sizes can often address rheumatic diseases and systemic inflammatory diseases \cite{2001_ScusselLonzetti_UpdatingAmericanCollege,2006_Cutolo_Nailfoldcapillaroscopyis}
	\item promising tool for vasculature structural imaging,1–3 breast tumor detection,4 epidermal melanin measurement,5,6 and oxygenation monitoring in blood vessels.7, see reference \cite{2006_Li_Improvedinvivo}
\end{itemize}

%%%%%%%%%%%%%%%%%%%%%%%%%%%%%%%%%%%%%%%%%%%%%%%%%%%%%%%%%%%%%%%%%%%%%%%%%%%%%%%%
\subsection{Synthetic Aperture}
\label{sec:SyntheticAperture} 


\cite{2012_Ma_Fastscanningcoaxial} 


 











