%see files examples for examples...



%% Set Language %%%%%%%%%%%%%%%%%%%%%%%%%%%%%%%%%%%%%%%%%%%%%%%%%%%%%%%%%%%%%%%%
\usepackage[english]{babel}
\usepackage[T1]{fontenc}
%\usepackage[ansinew]{inputenc}

%%\usepackage[ngerman]{babel}
%%\usepackage[T1]{fontenc}
%%\usepackage[ansinew]{inputenc}

%% Fonts %%%%%%%%%%%%%%%%%%%%%%%%%%%%%%%%%%%%%%%%%%%%%%%%%%%%%%%%%%%%%%%%%%%%%%%
%DO NOT CHANGE ORDER OF THIS!---------------------------------------------------
\usepackage{lmodern} %enhanced versions of the Computer Modern fonts
\renewcommand*{\familydefault}{\sfdefault} %sans serif fonts
\usepackage{sfmath} %sans serif equations
%NOW YOU CAN DO WHAT YOU WANT AGAIN!------------------------------------

%% Typesetting %%%%%%%%%%%%%%%%%%%%%%%%%%%%%%%%%%%%%%%%%%%%%%%%%%%%%%%%%%%%%%%%%
\usepackage{epigraph} %good looking quotes
\usepackage{eurosym} % have a guess :-)
\usepackage{units} %setting units in a typographically correct way
\usepackage{upgreek} %upright typesetting for greek symbols (eg. \upmu and \Upmu
\usepackage[version=3]{mhchem} %use of subscripts within the text environment to typeset chemical formulas, use like \ce{H2O}
%\usepackage{nicefrac} %nice fractions


%% General Stuff %%%%%%%%%%%%%%%%%%%%%%%%%%%%%%%%%%%%%%%%%%%%%%%%%%%%%%%%%%%%%%%

\usepackage{amsmath} % AMS Math Package
\usepackage{amsthm} % Theorem Formatting
\usepackage{amssymb}	% Math symbols such as \mathbb
\usepackage{multicol} % Allows for multiple columns

\usepackage{booktabs} %use \toprule, \bottomrule & \midrule in tables
\usepackage[section]{placeins} %Keep figures within section.
\usepackage{longtable}
\usepackage{array} 

%\usepackage[draft]{graphicx} %l�dt nur Platzhalter
\usepackage{graphicx} %%Zum Laden von Grafiken
\usepackage{wrapfig} %%Graphics with text wrpping around
\usepackage{subcaption}  %Placing figures/tables side-by-side,
\usepackage{caption} %???

%\usepackage[backend=bibtex, sorting=none, style=numeric, natbib=true, sortlocale=en_US, url=false, doi=true, eprint=false]{biblatex}
%\addbibresource{tex/bibo.bib}

\usepackage{url}	%creates clickable urls by entering \url{http://www.url.de}
%\usepackage{doi}	%hy­per­link to the tar­get of the DOI.

\usepackage[dvips,a4paper,margin=0.75in,bottom=1 in]{geometry}

%\usepackage{paralist} % very flexible & customisable lists (eg. enumerate/itemize, etc.)
%\usepackage{subcaption} 
%\usepackage[lofdepth,lotdepth]{subfig}include the subs in the listoffigures and listoftables
%\usepackage[cmyk]{xcolor} %convert all to CMYK

%% General Page Layout %%%%%%%%%%%%%%%%%%%%%%%%%%%%%%%%%%%%%%%%%%%%%%%%%%%%%%%%%

%\usepackage[left=20mm,					% linker Seitenrand f�rs Binden
						%right=20mm,					% rechter Seitenrand
						%top=25mm,					%	oberer Rand
						%bottom=25mm,				% unterer Rand
						%footskip=15mm,			% frei f�r Fu�noten
						%lines=34,					% Rumpf Zeilenauslastung
						%]{geometry}	

%% Header, Fooder, Chapter Headings and such %%%%%%%%%%%%%%%%%%%%%%%%%%%%%%%%%%%
\usepackage{fancyhdr}
\fancyhead{}  % Delete current header settings
\fancyfoot{}  % Delete current footer settings         

% pages and right on odd pages
\fancyhead[L]{\nouppercase{\leftmark}}      % Chapter in the right on even pages
%\fancyhead[R]{\nouppercase{\rightmark}}     % Section in the left on odd pages

\fancyfoot[C]{\thepage}  % Page number (boldface) centered everywhere

\renewcommand{\headwidth}{\textwidth} %set header width to text  width

%\usepackage[Lenny]{fncychap} %fancy chapter headings						

%% Special Tricks %%%%%%%%%%%%%%%%%%%%%%%%%%%%%%%%%%%%%%%%%%%%%%%%%%%%%%%%%%%%%

% Line break after \paragraph{Paragraph Name}						
\makeatletter
\renewcommand\paragraph{\@startsection{paragraph}{4}{\z@}%
  {-3.25ex\@plus -1ex \@minus -.2ex}%
  {1.5ex \@plus .2ex}%
  {\normalfont\normalsize\bfseries}}
\makeatother

% Kreis um Zahl, use as \myCircle{Zahl} %%%%%%%%%%%%%%%%%%%%%%%%%%%%%%%%%%%%%%%%
\newcommand{\myCircle}[1]{ \unitlength1ex\begin{picture}(2.5,2.5)%
\put(0.75,0.75){\circle{2.5}}\put(0.75,0.75){\makebox(0,0){#1}}\end{picture}} 

%% Source code printer for LATEX - Code direct in the .tex file %%%%%%%%%%%%%%%
\usepackage{listings} %highlighting of all the most common languages 
\usepackage{color}
 
\definecolor{dkgreen}{rgb}{0,0.6,0}
\definecolor{gray}{rgb}{0.5,0.5,0.5}
\definecolor{mauve}{rgb}{0.58,0,0.82}
 
\lstset{ %
  language=Mathematica,                % the language of the code
  basicstyle=\footnotesize,           % the size of the fonts that are used for the code
  numbers=left,                   % where to put the line-numbers
  numberstyle=\footnotesize,          % the size of the fonts that are used for the line-numbers
  stepnumber=1,                   % the step between two line-numbers. If it's 1, each line 
                                  % will be numbered
  numbersep=5pt,                  % how far the line-numbers are from the code
  backgroundcolor=\color{white},      % choose the background color. You must add \usepackage{color}
  showspaces=false,               % show spaces adding particular underscores
  showstringspaces=false,         % underline spaces within strings
  showtabs=false,                 % show tabs within strings adding particular underscores
  %frame=single,                   % adds a frame around the code
  tabsize=2,                      % sets default tabsize to 2 spaces
  captionpos=b,                   % sets the caption-position to bottom
  breaklines=true,                % sets automatic line breaking
  breakatwhitespace=false,        % sets if automatic breaks should only happen at whitespace
  title=\lstname,                   % show the filename of files included with \lstinputlisting;
                                  % also try caption instead of title
  numberstyle=\tiny\color{gray},        % line number style
  keywordstyle=\color{blue},          % keyword style
  commentstyle=\color{dkgreen},       % comment style
  stringstyle=\color{mauve},         % string literal style
  escapeinside={\%*}{*)},            % add a comment within your code
  morekeywords={*,ParallelTable, ParallelMap, Blur, ImageResize, ColorConvert, ImageDimensions, ImageData, ImageTake, ErrorListPlot} % if you want to add more keywords to the set
}



